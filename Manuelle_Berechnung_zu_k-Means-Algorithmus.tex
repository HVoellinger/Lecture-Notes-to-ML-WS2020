\documentclass{article}
\usepackage{amsmath}

\begin{document}

\section*{K-means Beispiel mit \( k = 3 \)}

Angenommen, wir haben die folgenden fünf Datenpunkte in einem zweidimensionalen Raum:

\[
\text{Punkte} = \{(2, 3), (3, 3), (6, 7), (8, 8), (10, 10)\}
\]

Wir möchten diese Punkte in \( k = 3 \) Cluster gruppieren.

\subsection*{Schritt 1: Initialisierung der Zentroiden}
Wir wählen zunächst zufällig drei Datenpunkte als Anfangszentroiden:

\begin{itemize}
    \item Zentroid 1: \( C_1 = (2, 3) \)
    \item Zentroid 2: \( C_2 = (6, 7) \)
    \item Zentroid 3: \( C_3 = (8, 8) \)
\end{itemize}

\subsection*{Schritt 2: Berechnung der Zuordnung}
Nun berechnen wir die euklidischen Distanzen jedes Datenpunkts zu den drei Zentroiden und weisen jeden Punkt dem nächstgelegenen Zentroiden zu.

Die euklidische Distanz zwischen zwei Punkten \( (x_1, y_1) \) und \( (x_2, y_2) \) ist gegeben durch:

\[
\text{Distanz} = \sqrt{(x_2 - x_1)^2 + (y_2 - y_1)^2}
\]

\textbf{Punkt (2, 3):}
\begin{align*}
\text{Distanz zu } C_1 = (2, 3): & \quad \sqrt{(2-2)^2 + (3-3)^2} = 0 \\
\text{Distanz zu } C_2 = (6, 7): & \quad \sqrt{(6-2)^2 + (7-3)^2} = \sqrt{16 + 16} = \sqrt{32} \approx 5{,}66 \\
\text{Distanz zu } C_3 = (8, 8): & \quad \sqrt{(8-2)^2 + (8-3)^2} = \sqrt{36 + 25} = \sqrt{61} \approx 7{,}81
\end{align*}

Nächstgelegener Zentroid: \( C_1 \)

\textbf{Punkt (3, 3):}
\begin{align*}
\text{Distanz zu } C_1 = (2, 3): & \quad \sqrt{(3-2)^2 + (3-3)^2} = \sqrt{1} = 1 \\
\text{Distanz zu } C_2 = (6, 7): & \quad \sqrt{(6-3)^2 + (7-3)^2} = \sqrt{9 + 16} = \sqrt{25} = 5 \\
\text{Distanz zu } C_3 = (8, 8): & \quad \sqrt{(8-3)^2 + (8-3)^2} = \sqrt{25 + 25} = \sqrt{50} \approx 7{,}07
\end{align*}

Nächstgelegener Zentroid: \( C_1 \)

\textbf{Punkt (6, 7):}
\begin{align*}
\text{Distanz zu } C_1 = (2, 3): & \quad \sqrt{(6-2)^2 + (7-3)^2} = \sqrt{16 + 16} = \sqrt{32} \approx 5{,}66 \\
\text{Distanz zu } C_2 = (6, 7): & \quad \sqrt{(6-6)^2 + (7-7)^2} = 0 \\
\text{Distanz zu } C_3 = (8, 8): & \quad \sqrt{(8-6)^2 + (8-7)^2} = \sqrt{4 + 1} = \sqrt{5} \approx 2{,}24
\end{align*}

Nächstgelegener Zentroid: \( C_2 \)

\textbf{Punkt (8, 8):}
\begin{align*}
\text{Distanz zu } C_1 = (2, 3): & \quad \sqrt{(8-2)^2 + (8-3)^2} = \sqrt{36 + 25} = \sqrt{61} \approx 7{,}81 \\
\text{Distanz zu } C_2 = (6, 7): & \quad \sqrt{(8-6)^2 + (8-7)^2} = \sqrt{4 + 1} = \sqrt{5} \approx 2{,}24 \\
\text{Distanz zu } C_3 = (8, 8): & \quad \sqrt{(8-8)^2 + (8-8)^2} = 0
\end{align*}

Nächstgelegener Zentroid: \( C_3 \)

\textbf{Punkt (10, 10):}
\begin{align*}
\text{Distanz zu } C_1 = (2, 3): & \quad \sqrt{(10-2)^2 + (10-3)^2} = \sqrt{64 + 49} = \sqrt{113} \approx 10{,}63 \\
\text{Distanz zu } C_2 = (6, 7): & \quad \sqrt{(10-6)^2 + (10-7)^2} = \sqrt{16 + 9} = \sqrt{25} = 5 \\
\text{Distanz zu } C_3 = (8, 8): & \quad \sqrt{(10-8)^2 + (10-8)^2} = \sqrt{4 + 4} = \sqrt{8} \approx 2{,}83
\end{align*}

Nächstgelegener Zentroid: \( C_3 \)

\subsection*{Zuordnung nach dem ersten Schritt}
\begin{itemize}
    \item \( C_1 \): \{(2, 3), (3, 3)\}
    \item \( C_2 \): \{(6, 7)\}
    \item \( C_3 \): \{(8, 8), (10, 10)\}
\end{itemize}

\subsection*{Schritt 3: Zentroiden aktualisieren}
Wir berechnen die neuen Zentroiden, indem wir den Mittelwert der Punkte in jedem Cluster berechnen.

\begin{itemize}
    \item Neuer Zentroid für \( C_1 \): Mittelwert von \{(2, 3), (3, 3)\} = \(\left(\frac{2+3}{2}, \frac{3+3}{2}\right) = (2.5, 3)\)
    \item Neuer Zentroid für \( C_2 \): Da es nur einen Punkt gibt, bleibt der Zentroid gleich \( (6, 7) \).
    \item Neuer Zentroid für \( C_3 \): Mittelwert von \{(8, 8), (10, 10)\} = \(\left(\frac{8+10}{2}, \frac{8+10}{2}\right) = (9, 9)\)
\end{itemize}

\subsection*{Schritt 4: Wiederholung}
Wir würden nun die Schritte 2 und 3 wiederholen, indem wir die Zuordnung der Punkte basierend auf den aktualisierten Zentroiden erneut berechnen und die Zentroiden so lange aktualisieren, bis sich die Zuordnungen nicht mehr ändern.

\subsection*{Endergebnis}
Nach mehreren Iterationen (in diesem einfachen Beispiel könnten bereits 2 Iterationen ausreichen) würden wir eine stabile Zuordnung und endgültige Zentroiden haben. Das Ergebnis wären die 3 Cluster mit den zugehörigen Punkten.

\end{document}