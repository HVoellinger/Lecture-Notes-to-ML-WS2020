\documentclass{article}
\usepackage{amsmath}
\usepackage{amsfonts}
\usepackage{amssymb}

\begin{document}

\section*{3-Means-Algorithmus}

Gegeben seien die folgenden 8 Punkte in einem zweidimensionalen Raum:
\[
\text{Punkt A}: (1, 1), \quad \text{Punkt B}: (2, 1), \quad \text{Punkt C}: (4, 3), \quad \text{Punkt D}: (5, 4), \quad
\]
\[
\text{Punkt E}: (6, 7), \quad \text{Punkt F}: (8, 8), \quad \text{Punkt G}: (3, 3), \quad \text{Punkt H}: (7, 7)
\]

\subsection*{Schritt 1: Initialisierung der Cluster-Zentren}

Wähle 3 Anfangszentren zufällig aus den Punkten:
\[
\text{Cluster 1 Zentrum}: (1, 1), \quad \text{Cluster 2 Zentrum}: (5, 4), \quad \text{Cluster 3 Zentrum}: (8, 8)
\]

\subsection*{Schritt 2: Zuordnung der Punkte zu den Zentren}

Für jeden Punkt \( P_i \) berechne die euklidische Distanz zu jedem Zentrum \( C_j \) und ordne den Punkt dem nächstgelegenen Zentrum zu:
\[
d(P_i, C_j) = \sqrt{(x_i - x_j)^2 + (y_i - y_j)^2}
\]

\subsection*{Zuordnungsergebnisse:}

\begin{itemize}
    \item \textbf{Cluster 1:} Punkte A (1, 1), B (2, 1)
    \item \textbf{Cluster 2:} Punkte C (4, 3), D (5, 4), G (3, 3)
    \item \textbf{Cluster 3:} Punkte E (6, 7), F (8, 8), H (7, 7)
\end{itemize}

\subsection*{Schritt 3: Aktualisierung der Cluster-Zentren}

Berechne die neuen Cluster-Zentren als Mittelwert der Punkte in jedem Cluster:
\[
\text{Neues Zentrum von Cluster 1}: \left( \frac{1+2}{2}, \frac{1+1}{2} \right) = (1.5, 1)
\]
\[
\text{Neues Zentrum von Cluster 2}: \left( \frac{4+5+3}{3}, \frac{3+4+3}{3} \right) = (4, 3.33)
\]
\[
\text{Neues Zentrum von Cluster 3}: \left( \frac{6+8+7}{3}, \frac{7+8+7}{3} \right) = (7, 7.33)
\]

\subsection*{Schritt 4: Wiederholung der Schritte 2 und 3}

Wiederhole die Schritte 2 und 3, bis sich die Cluster-Zentren nicht mehr signifikant ändern. In diesem Beispiel konvergiert der Algorithmus nach 2 Iterationen.

\subsection*{Endergebnis:}
Die endgültigen Cluster-Zentren sind:
\[
\text{Cluster 1 Zentrum}: (1.5, 1), \quad \text{Cluster 2 Zentrum}: (4, 3.33), \quad \text{Cluster 3 Zentrum}: (7, 7.33)
\]
Die 8 Punkte sind in 3 Gruppen aufgeteilt. Es waren 2 Iterationen des Algorithmus notwendig.
\end{document}