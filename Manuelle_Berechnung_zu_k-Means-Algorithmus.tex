\documentclass{article}
\usepackage{amsmath}
\usepackage{amsfonts}
\usepackage{amssymb}
\usepackage{tikz}

\begin{document}

\title{Beispiel für den 3-Means-Algorithmus}
\author{Dr. Hermann Völlinger}
\date{01.09.2024}
\maketitle 

\section*{Ausgangsdaten}
Wir haben die folgenden acht Punkte in einem 2D-Raum:
\begin{align*}
\text{Punkt A:} & \quad (1, 1) \\
\text{Punkt B:} & \quad (2, 1) \\
\text{Punkt C:} & \quad (4, 3) \\
\text{Punkt D:} & \quad (5, 4) \\
\text{Punkt E:} & \quad (6, 7) \\
\text{Punkt F:} & \quad (8, 8) \\
\text{Punkt G:} & \quad (3, 3) \\
\text{Punkt H:} & \quad (7, 7)
\end{align*}

Ziel ist es, diese Punkte in \textbf{drei Cluster} aufzuteilen.

\section*{Schritt 1: Initialisierung der Cluster-Zentren}
Wir wählen drei Anfangszentren zufällig aus den Punkten:
\begin{align*}
\text{Cluster 1 Zentrum:} & \quad (1, 1) \quad \text{(Punkt A)} \\
\text{Cluster 2 Zentrum:} & \quad (5, 4) \quad \text{(Punkt D)} \\
\text{Cluster 3 Zentrum:} & \quad (8, 8) \quad \text{(Punkt F)}
\end{align*}

\section*{Iteration 1: Zuordnung der Punkte zu den Zentren}
Für jeden Punkt berechnen wir die euklidische Distanz zu den drei Zentren und ordnen den Punkt dem nächstgelegenen Zentrum zu.

\subsection*{Berechnung der Distanzen und Zuordnung}

\begin{itemize}
    \item \textbf{Punkt A (1, 1):}
    \begin{align*}
    d(A, \text{Cluster 1}) &= \sqrt{(1-1)^2 + (1-1)^2} = 0 \\
    d(A, \text{Cluster 2}) &= \sqrt{(1-5)^2 + (1-4)^2} = \sqrt{16 + 9} = 5 \\
    d(A, \text{Cluster 3}) &= \sqrt{(1-8)^2 + (1-8)^2} = \sqrt{49 + 49} = 9.899 \\
    \textbf{Zuordnung:} & \quad \text{Cluster 1}
    \end{align*}
    
    \item \textbf{Punkt B (2, 1):}
    \begin{align*}
    d(B, \text{Cluster 1}) &= \sqrt{(2-1)^2 + (1-1)^2} = \sqrt{1} = 1 \\
    d(B, \text{Cluster 2}) &= \sqrt{(2-5)^2 + (1-4)^2} = \sqrt{9 + 9} = 4.243 \\
    d(B, \text{Cluster 3}) &= \sqrt{(2-8)^2 + (1-8)^2} = \sqrt{36 + 49} = 9.219 \\
    \textbf{Zuordnung:} & \quad \text{Cluster 1}
    \end{align*}
    
    \item \textbf{Punkt C (4, 3):}
    \begin{align*}
    d(C, \text{Cluster 1}) &= \sqrt{(4-1)^2 + (3-1)^2} = \sqrt{9 + 4} = 3.606 \\
    d(C, \text{Cluster 2}) &= \sqrt{(4-5)^2 + (3-4)^2} = \sqrt{1 + 1} = 1.414 \\
    d(C, \text{Cluster 3}) &= \sqrt{(4-8)^2 + (3-8)^2} = \sqrt{16 + 25} = 6.403 \\
    \textbf{Zuordnung:} & \quad \text{Cluster 2}
    \end{align*}
    
    \item \textbf{Punkt D (5, 4):}
    \begin{align*}
    d(D, \text{Cluster 1}) &= \sqrt{(5-1)^2 + (4-1)^2} = \sqrt{16 + 9} = 5 \\
    d(D, \text{Cluster 2}) &= \sqrt{(5-5)^2 + (4-4)^2} = 0 \\
    d(D, \text{Cluster 3}) &= \sqrt{(5-8)^2 + (4-8)^2} = \sqrt{9 + 16} = 5 \\
    \textbf{Zuordnung:} & \quad \text{Cluster 2}
    \end{align*}
    
    \item \textbf{Punkt E (6, 7):}
    \begin{align*}
    d(E, \text{Cluster 1}) &= \sqrt{(6-1)^2 + (7-1)^2} = \sqrt{25 + 36} = 7.810 \\
    d(E, \text{Cluster 2}) &= \sqrt{(6-5)^2 + (7-4)^2} = \sqrt{1 + 9} = 3.162 \\
    d(E, \text{Cluster 3}) &= \sqrt{(6-8)^2 + (7-8)^2} = \sqrt{4 + 1} = 2.236 \\
    \textbf{Zuordnung:} & \quad \text{Cluster 3}
    \end{align*}
    
    \item \textbf{Punkt F (8, 8):}
    \begin{align*}
    d(F, \text{Cluster 1}) &= \sqrt{(8-1)^2 + (8-1)^2} = \sqrt{49 + 49} = 9.899 \\
    d(F, \text{Cluster 2}) &= \sqrt{(8-5)^2 + (8-4)^2} = \sqrt{9 + 16} = 5 \\
    d(F, \text{Cluster 3}) &= \sqrt{(8-8)^2 + (8-8)^2} = 0 \\
    \textbf{Zuordnung:} & \quad \text{Cluster 3}
    \end{align*}
    
    \item \textbf{Punkt G (3, 3):}
    \begin{align*}
    d(G, \text{Cluster 1}) &= \sqrt{(3-1)^2 + (3-1)^2} = \sqrt{4 + 4} = 2.828 \\
    d(G, \text{Cluster 2}) &= \sqrt{(3-5)^2 + (3-4)^2} = \sqrt{4 + 1} = 2.236 \\
    d(G, \text{Cluster 3}) &= \sqrt{(3-8)^2 + (3-8)^2} = \sqrt{25 + 25} = 7.071 \\
    \textbf{Zuordnung:} & \quad \text{Cluster 2}
    \end{align*}
    
    \item \textbf{Punkt H (7, 7):}
    \begin{align*}
    d(H, \text{Cluster 1}) &= \sqrt{(7-1)^2 + (7-1)^2} = \sqrt{36 + 36} = 8.485 \\
    d(H, \text{Cluster 2}) &= \sqrt{(7-5)^2 + (7-4)^2} = \sqrt{4 + 9} = 3.605 \\
    d(H, \text{Cluster 3}) &= \sqrt{(7-8)^2 + (7-8)^2} = \sqrt{1 + 1} = 1.414 \\
    \textbf{Zuordnung:} & \quad \text{Cluster 3}
    \end{align*}
\end{itemize}

\subsection*{Ergebnis der ersten Iteration}
\begin{itemize}
    \item \textbf{Cluster 1:} Punkte A, B
    \item \textbf{Cluster 2:} Punkte C, D, G
    \item \textbf{Cluster 3:} Punkte E, F, H
\end{itemize}

\section*{Schritt 3: Neuberechnung der Cluster-Zentren}
\begin{itemize}
    \item \textbf{Neues Zentrum für Cluster 1:} \((\frac{1+2}{2}, \frac{1+1}{2}) = (1.5, 1)\)
    \item \textbf{Neues Zentrum für Cluster 2:} \((\frac{4+5+3}{3}, \frac{3+4+3}{3}) = (4, 3.33)\)
    \item \textbf{Neues Zentrum für Cluster 3:} \((\frac{6+8+7}{3}, \frac{7+8+7}{3}) = (7, 7.33)\)
\end{itemize}

\section*{Iteration 2: Neue Zuordnung der Punkte}
Nach der Neuberechnung der Cluster-Zentren bleiben die Punkte den gleichen Clustern zugeordnet. Daher ist der Algorithmus nach zwei Iterationen konvergiert.

\subsection*{Endergebnis}
Die finalen Cluster sind:
\begin{itemize}
    \item \textbf{Cluster 1 (Zentrum: (1.5, 1)):} Punkte A, B
    \item \textbf{Cluster 2 (Zentrum: (4, 3.33)):} Punkte C, D, G
    \item \textbf{Cluster 3 (Zentrum: (7, 7.33)):} Punkte E, F, H
\end{itemize}

\end{document}